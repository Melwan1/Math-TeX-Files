\documentclass[12pt,a4paper]{report}
\input{00 - preambule}

\begin{document}

\chapter{Développements limités}

\section{Définitions et exemples}
\begin{definition}{Développement limité}{}
Soit $I$ un intervalle de $\R$, $f:I\to \K$, $a\in I$, $n \in \N$\\
On dit que $f$ admet un développement limité à l'ordre $n$ au voisinage de $a$ (\textit{en abrégé}: $f$ admet un $DL_n(a)$) s'il existe $P\in \K[X]$ de degré $\leq n$ tel que : 
\begin{center}
    $f(x)-P(x-a) \underset{x \to a}{=} o((x-a)^n)$ 
\end{center}
\ie s'il existe $\lambda_0$, ..., $\lambda_n$ $\in K$ tel que :

\begin{center}
    $f(x)-(\lambda_0+\lambda_1(x-a)+...+\lambda_n(x-a)^n)\underset{x \to a}{=}o((x-a)^n)$    
\end{center}
On écrit alors : au $\mathcal{V}(a)$, $\mathbox{f(x)=\underbrace{\lambda_0+...+\lambda_n(x-a)^n}_{\text{Partie polynomiale}} + \underbrace{o((x-a)^n)}_{\text{\textit{chouia}}}}$
\end{definition}

\begin{remarque}
\begin{itemize}
	\item Si $f$ admet un $DL_n(a)$, il y a \strong{un seul} $P \in \K[X]$ de degré $\le n$ tel que, au voisinage de $a$ : $f(x) = P(x-a)+ o((x-a)^n)$.
	$x \mapsto P(x-a)$ s'appelle la \strong{partie régulière} du $DL_n(a)$.

	\begin{demo}
	Supposons que, au voisinage de $a$, $f(x) = P(x-a) + o((x-a)^n) = Q(x-a) + o((x-a)^n)$. avec 	$P,Q$ polynômes de degré $\le n$. Alors $P(x-a) - Q(X-a) \underset{x \to a}{=} o((x-a)^n)$. \\
	Ecrivons $P-Q = \lambda_0 + ... + \lambda n X^n$. Supposons : \\
	$\exists k \in \llbracket 0,n \rrbracket, \lambda_k \neq 0$. Soit $\strong{m = \min \{ k \in \llbracket 0,n \rrbracket \mid \lambda_k \neq 0 \}}$. \\ \\
	D'où $(P-Q)(x-a) = \lambda_m (x-a)^m + \underbrace{\lambda_{m+1}(x-a)^{m+1} + ... + \lambda_n (x-a)^n}_{\underset{x \to a}{=} o((x-a)^m)}$. \\
	D'où $(P-Q)(x-a) \underset{x \to a}{\sim} (x-a)^m$ et on aurait $\lambda_m (x-a)^m \underset{x \to a}{=} o((x-a)^m)$, c'est faux.
	\end{demo}
	
	\item On peut toujours supposer $a=0$. \\
	$f : I \to \K, n \in \N, a \in I$. $J = I-a = \{x-a \mid x \in I \}$. \\
	Soit $g : t \in J \mapsto f(a+t) \in \K$, on a $0 \in J$. \\
	Soit $P \in \K[X]$ de degré $\le n$. $f$ admet un $DL_n(a)$ de partie régulière $x \mapsto P(x-a) \\
	\Longleftrightarrow g$ admet un $DL_n(0)$ de partie régulière $t \mapsto P(t)$.
	En effet, supposons que, au voisinage de $a$, $f(x) = P(x-a) + (x-a)^m \varepsilon(x)$ avec $\underset{x \to a}{\lim} \varepsilon(x) = 0$. \\
	Alors, pour $t \in J$, $g(t) = f(a+t) = \underbrace{P(t)}_{\text{polynomiale de degré } \le n} + t^n \underbrace{\varepsilon{a+t}}_{\xrightarrow[t \to 0]{} 0}$.
	
	\item $f : I \to \K, a \in I$. $f$ a un $DL_0(a)$ signifie $f$ est continue en $a$. \\
	$\Longrightarrow$ Au voisinage de $a$, $f(x) = \lambda + \varepsilon(x)$ avec $\underset{a}{\lim} \varepsilon = 0$. \\
	D'où $f(x) \xrightarrow[x \to a]{} \lambda$ (d'où $\lambda = f(a)$) \\
	$\Longleftarrow$ On écrit $f(x) = \underbrace{f(a)}_{\substack{\text{polynomiale} \\ \text{de degré } 0}} + \underbrace{f(x)-f(a)}_{\varepsilon(x) \xrightarrow[x \to a]{} 0}$. \\ \\
	$f$ a un $DL_1(a) \Longleftrightarrow f$ est dérivable en $a$. Si $f$ est dérivable en $a$, on a au voisinage de $a$, $f(x) = f(a) + (x-a)f'(a) + o(x-a)$. \\ \\
	\textbf{Attention !} On n'a pas pour $n \ge 2$, $f$ admet un $DL_n(a) \Longleftrightarrow f$ a une dérivée $n^{\text{ième}}$ en $a$.
\end{itemize}
\end{remarque}

\begin{exemple}
$f(x) = \begin{cases}
0 &\text{si } x=0 \\
x^3\sin\left(\dfrac{1}{x}\right) &\text{si } x \neq 0
\end{cases}$ \\ \\
$f$ est de classe $\mathcal{C}^{\infty}$ sur $\R^*$. \\ \\
Pour $x \neq 0$, $\dfrac{f(x)-f(0)}{x} = x^2\sin\left(\dfrac{1}{x}\right) \xrightarrow[x \to 0]{} 0$ donc $f$ est dérivable en $0$ et $f'(0) = 0$. \\ \\
Pour $x \neq 0$, $\dfrac{f'(x)-f'(0)}{x} = \dfrac{f'(x)}{x} = \dfrac{3x^2\sin \frac{1}{x} - x \cos \frac{1}{x}}{x} = \underbrace{3x \sin \dfrac{1}{x}}_{\xrightarrow[x \to 0]{} 0} - \underbrace{\cos \dfrac{1}{x}}_{\text{pas de limite en } 0}$, donc $f'$ n'est pas dérivable en $0$.
\end{exemple}

\begin{exemple}
$f : x \in ]-\infty,1[ \mapsto \dfrac{1}{1-x}$. \\ \\
Alors, $\forall n \in \N, f$ admet un $DL_n(0)$ de partie régulière $x \mapsto 1+x+...+x^n$. \\ \\
En effet, pour $x \in ]-\infty,1[, 1+x+...+x^n = \dfrac{1-x^{n+1}}{1-x}$ \ie $f(x) = \underbrace{1+x+...+x^n}_{\text{polynôme de degré } \le n} + \underbrace{\dfrac{x}{1-x} x^n}_{\underset{x \to 0}{=} o(x^n)}$. \\ \\
Donc, au voisinage de $0$, $\dfrac{1}{1-x} = 1+x+...x^n + o(x^n)$.
\end{exemple}

\begin{theoreme}{Formule de \textsc{Taylor-Young}\footnotemark{}}{FormuleTaylorYoung}
    Soit $n \in \N$, $I$ un \strong{intervalle} de $\R$, $f : I \to \K$ de classe \strong{$\mathcal{C}^n$} et \strong{$a \in I$}. Alors $f$ admet un $\mathrm{DL}_n(a)$ dont la partie régulière est :
    $$ x \mapsto T_{n,f,a} (x) = \sum_{k = 0}^n f^{(k)}(a) \dfrac{(x - a)^k}{k!} = f(a) + f'(a)(x-a) + ... + f^{(n)}(a)\dfrac{(x-a)^n}{n!} $$
    Ainsi, au voisinage de $a$, $f$ s'écrit :
    $$ \mathbox{f(x) \underset{x \to a}{=} \sum_{k = 0}^n f^{(k)}(a) \dfrac{(x - a)^k}{k!} + o\left( (x - a)^n \right)} $$
\end{theoreme}

\footnotetext{\textit{Aka} Taylor Le Jeune}

\begin{principedemo}{FormuleTaylorYoung}
    Raisonner par récurrence pour montrer que pour $n \in \N$ :
    $$ \strong{\dfrac{f(x) - T_{n,f,a}(x)}{(x-a)^n} \xrightarrow[x \to a]{} 0} $$
    Utiliser alors la \strong{définition epsilonesque} de la limite et les \strong{IAF} entre $a$ et $x$ sur les fonctions :
    $$ \strong{\varphi : t \mapsto f(t) - T_{n+1,f,a}(t)} \quad\text{et}\quad \strong{g : t \mapsto \varepsilon \dfrac{|t-a|^{n+1}}{n+1} } $$
    Distinguer les cas \strong{$x < a$} et \strong{$x > a$} pour gérer plus facilement les valeurs absolues.
\end{principedemo}

\begin{remarque}
    Si $f$ admet un $\mathrm{DL}_n(a)$ de partie régulière $T_{n,f,a}$, alors au voisinage de $0$, $f$ s'écrit :
    $$ f(x) \underset{x \to 0}{=} T_{n,f,a}(x + a) + o\left(x^n\right) $$
    De même, si $f$ admet un $\mathrm{DL}_n(0)$ de partie régulière $T_{n,f,0}$, alors au voisinage de $a$, $f$ s'écrit :
    $$ f(x) \underset{x \to a}{=} T_{n,f,0}(x - a) + o\left((x-a)^n\right) $$
\end{remarque}

Au $\mathcal{V}(0)$ : $f(a+h)=f(a)+hf'(a)+...+\dfrac{h^n}{n!} f^{(n)}(a)+o(h^n)$

\begin{proposition}{$DL$ classiques}{}
Au $\mathcal{V}(0)$:
\begin{itemize}
    \item $\sin x = x - \dfrac{x^3}{6}+\dfrac{x^5}{120}+...+(-1)^n \dfrac{x^{2n+1}}{(2n+1)!}+o(x^{2n+1})$
    \item $\cos x = 1 - \dfrac{x^2}{2}+\dfrac{x^4}{24}-...+(-1)^n \dfrac{x^{2n}}{(2n)!}+o(x^{2n})$
    \item $e^x = 1+x +\dfrac{x^2}{2}+...+ \dfrac{x^{n}}{n!}+o(x^{n})$
    \item $\ln(1+x) = x - \dfrac{x^2}{2}+\dfrac{x^3}{3}+...+ (-1)^{n-1}\dfrac{x^{n}}{n}+o(x^{n})$
    \item $(1+x)^{\alpha} = 1+ \alpha x + \dfrac{\alpha (\alpha-1)}{2} x^2+...+ \dfrac{\alpha (\alpha-1)...(\alpha-n+1)}{n!} x^n+o(x^{n})$
\end{itemize}
\end{proposition}

\section{Opérations sur les $DL$}
On se place désormais au $\mathcal{V} (0)$

\subsection{Produit élémentaire}
\begin{proposition}{Produit élémentaire}{}
$f:I\to \K$ ($0\in I$)
On suppose : $f$ admet un $DL_n (0)$ de partie régulière $P\in \K[X]$, alors :
\begin{center}
    \strong{$x\mapsto x^m f(x)$ admet un $DL_{n+m}(0)$ de partie régulière $X^m P$}
\end{center}
\end{proposition}

\begin{demo}
En effet, au $\mathcal{V}(0)$, $f(x)=P(x)+o(x^n)$\\
D'où : $x^m f(x) = \underbrace{x^m P(x)}_{\text{Polynomiale de degré } \leq n+m} + \underbrace{x^m o(x^n)}_{o(x^{n+m})}$
\end{demo}

\subsection{Composition élémentaire}
\begin{proposition}{Composition élémentaire}{}
$f:I\to \K$ ($0\in I$)
On suppose : $f$ admet un $DL_n (0)$ de partie régulière $P\in \K[X]$ ($\deg P \leq n$), alors :
\begin{center}
    \strong{$x\mapsto f(\alpha x^m)$ admet un $DL_{nm}(0)$ de partie régulière $Q = P \circ (\alpha X^m)$}
\end{center}
\end{proposition}

\begin{demo}
Au $\mathcal{V}(0)$, $f(t)=P(t)+t^n\varepsilon(t)$ avec $\displaystyle\lim_{0} \varepsilon = 0$\\
D'où : $f(\alpha x^m ) = \underbrace{P(\alpha x^m)}_{\text{Polynomiale de degré } \leq nm} + \alpha^n x^{nm} \underbrace{ \varepsilon(\alpha x^m)}_{\xrightarrow[x \to 0]{} 0}$
\end{demo}

\subsection{Troncature}
\begin{proposition}{Troncature}{}
$f:I\to \K$ ($0\in I$) admet un $DL_n (0)$ de partie régulière $P\in \K[X]$, alors pour $0\leq k \leq n$,
\begin{center}
    $f$ admet un $DL_k (0)$ de partie régulière \textbox{$a_0 + a_1 X + ... + a_k X^k$} ($P$ tronqué au degré $k$)
\end{center}
\end{proposition}

\begin{demo}
Soit $k \in \llbracket 0, n \rrbracket$ au $\mathcal{V}(0)$\\
$f(x) = a_0 + ... + a_k x^k + \underbrace{a_{k+1} x^{k+1}}_{o(x^k)} + ... + \underbrace{a_{n} x^{n}}_{o(x^k)} + \underbrace{o(x^n)}_{o(x^k)}$\\
$f(x)=\underbrace{a_0 + ... + a_{k} x^{k}}_{\text{Polynomiale de degré } \leq k} + o(x^k)$
\end{demo}

\subsection{Parité}
\begin{proposition}{Parité}{}
$f:I\to \K$ ($I$ symétrique par rapport à 0) admet un $DL_n (0)$ de partie régulière $P\in \K[X]$ :
\begin{center}
    Si $f$ est paire alors $P$ est pair \\
    \ie $P=P(-x)$, si on écrit $P=\displaystyle\sum_k a_k X^k$, on a $a_k=0$ si $k$ est impair)\\
\end{center}
\begin{center}
    Si $f$ est impaire, $P$ est impair\\
    \ie $P(-x)=-P$ ou encore $a_k = 0$ si $k$ est pair
\end{center}
\end{proposition}

\begin{demo}
Supposons, par exemple, $f$ paire au voisinage de 0, $f(x)=P(x)+x^n \varepsilon(x)$ où $\displaystyle\lim_0 \varepsilon = 0$ 
\begin{center}
    $f(x)=f(-x)=P(-x)+(-x)^n \varepsilon (-x)$\\
    $f(x)=f(-x)=P(-x)+(-1)^n x^n \varepsilon (-x)$\\
    $f(x)= P(-x) + o(x^n)$
\end{center}
Par unicité de la partie régulière d'un $DL$ :
\begin{center}
    $P=P(-X)$ d'où $\displaystyle\sum_k a_k (-X)^k = \displaystyle\sum_k (-1)^k a_k X^k = \displaystyle\sum_k a_k X^k$
\end{center}
D'où, pour tout entier naturel $k$, $(-1)^k$ $a_k$ = $a_k$\\
Autrement dit : $a_k=0$ si $k$ impair
\end{demo}

\subsection{Combinaison linéaire}
\begin{proposition}{Combinaison linéaire}{}
$\alpha \in \K$, $f,g:I\to \K$ admettent des $DL_n (0)$ de parties régulières $P,Q \in \K[X]$, alors :
\begin{center}
    $\alpha f + g$ admet un $DL_n (0)$ de partie régulière $\alpha P + Q$
\end{center}
\end{proposition}

\begin{demo}
Au voisinage de 0 :
\begin{center}
    $f(x) = P(x) + o(x^n)$
    $g(x) = Q(x) + o(x^n)$
\end{center}
D'où $\alpha f(x) +g(x)= \underbrace{\alpha P(x) + Q(x)}_{\text{Polynomiale de degré } \leq n} + \underbrace{\alpha o(x^n) + o(x^n)}_{o(x^n)}$
\end{demo}

\begin{exemple}
\strong{$DL_2 (0)$ de $\sqrt{1+x} - e^x$}\\
\\
Au voisinage de 0 : 
\begin{center}
    $e^x = 1 +x+ \dfrac{x^2}{2} + o(x^2)$ \: \: \: \: \: $\sqrt{1+x} = 1 + \dfrac{x}{2} - \dfrac{x^2}{8} + o(x^2)$
\end{center}
D'où \textbox{$\sqrt{1+x} - e^x = - \dfrac{x}{2} - \dfrac{5}{8}x^2 +o(x^2) $}\\
\\
On trouve également un équivalent : 
\begin{center}
    $\sqrt{1+x} - e^x = - \dfrac{x}{2} +o(x)$\\
    \textbox{$\sqrt{1+x} - e^x \underset{x \to 0}{\sim}$ $- \dfrac{x}{2}$}
\end{center}
\end{exemple}

\pagebreak

\subsection{Produit}
\begin{proposition}{Produit}{}
$f,g : I \to \K$ ont des $DL_n (0)$ de parties régulières $P$,$Q \in \K[X]$, alors :
\begin{center}
    $fg$ a un $DL_n(0)$ dont la partie régulière est $[PQ]_n$ (polynôme $PQ$ tronqué au degré $n$)
\end{center}
\end{proposition}

\begin{demo}
\textit{A venir ...}
\end{demo}

\begin{exemple}
\strong{$DL_2 (0)$ de $\sqrt{1+x}$ $e^x$}\\
\\
Au voisinage de 0 : 
\begin{center}
    $e^x = 1 +x+ \dfrac{x^2}{2} + o(x^2)$ \: \: \: \: \: $\sqrt{1+x} = 1 + \dfrac{x}{2} - \dfrac{x^2}{8} + o(x^2)$\\
\end{center}
D'où : 
\begin{center}
    $\sqrt{1+x}$ $e^x = 1 + x(\dfrac{1}{2} + 1) + x^2 (- \dfrac{1}{8}+ \dfrac{1}{2} + \dfrac{1}{2}) + o(x^2)$\\
    \textbox{$\sqrt{1+x}$ $e^x = 1 + \dfrac{3}{2}x + \dfrac{7}{8} x^2 + o(x^2)$}
\end{center}
\end{exemple}

\begin{application}{Exercice}{}
\begin{align*}
  f \colon ]0,1] &\to \R\\
  x &\mapsto \dfrac{1}{\sin x} - \dfrac{1}{x}
\end{align*}
Montrer que $f$ se prolonge en 0, en une fonction de classe $C^1$
\end{application}

\subsection{Intégration}
\begin{proposition}{Intégration}{}
$f:I \to K$ dérivable\\
On suppose $f'$ admet un $DL_n (0)$ de partie régulière $P=a_0 +a_1 X + ... + a_n X^n$, alors :
\begin{center}
    $f$ a un $DL_{n+1} (0)$ de partie régulière \strong{$Q=f(0)+a_0 X + \dfrac{a_1}{2} X^2 + ... + \dfrac{a_n}{n+1} X^{n+1}$}
\end{center}
\end{proposition}

\begin{demo}
\textit{A venir ...}
\end{demo}

\begin{application}{}{}
\begin{align*}
  f \colon ]-1,+\infty] &\to \R\\
  x &\mapsto \ln (1+x)
\end{align*}
\end{application}

\subsection{Dérivation}
\begin{proposition}{Dérivation}{}
$f:I \to \K$ de classe $C^{n+1} (0 \in I)$\\
Soit $Q$ la partie régulière du $DL_{n+1} (0)$ de $f$ (on sait qu'un tel $DL_{n+1} (0)$ existe)\\
Soit $P$ la partie régulière du $DL_{n} (0)$ de $f'$ (existe car $f'$ est $C^n$)
\begin{center}
    $P=Q'$
\end{center}
\end{proposition}

\begin{demo}
On sait que $Q=T_{n+1,f,0}$ et $P=T_{n,f',0}$\\
Il s'agit de voir que : $T_{n,f',0}= (T_{n+1,f,0})'$ ce qui est vrai
\end{demo}

\begin{exemple}
\textit{A venir ...}
\end{exemple}

\subsection{Composition}
\begin{proposition}{Composition}{}
$g:I \to K$\\
$f:I \to J$ tel que \Strong{$f(0)=0$}\\
On suppose $g$ a un $DL_n (0)$ de partie régulière $Q \in \K[X]$ et que $f$ a un $DL_n (0)$ de partie régulière $P \in K[X]$\\
(On a $P(0)=f(0)=0$, donc $P$ s'écrit : $P=XS$ où $\deg S \leq n-1$), alors :
\begin{center}
    $g\circ f$ admet un $DL_n(0)$ dont la partie régulière est $[Q \circ P]_n$ ($Q\circ P$ tronqué au degré $n$)
\end{center}
\end{proposition}

\begin{demo}
\textit{A venir ...}
\end{demo}

\pagebreak

\subsection{Inverse}
\begin{proposition}{Inverse}{}
$f:I\to \K$ ($0 \in I$)\\
On suppose $f(0) \neq 0$ et $f$ admet un $DL_n (0)$ ($f$ est continue en 0, $f(0) \neq 0$ donc $f$ reste $\neq 0$ au $\mathcal{V}(0)$ donc $\dfrac{1}{f}$ est définie au $\mathcal{V}(0)$), alors : 
\begin{center}
    $\dfrac{1}{f}$ admet un $DL_n (0)$
\end{center}
\end{proposition}

\pagebreak

\section*{Démonstrations}
\addcontentsline{toc}{section}{Démonstrations}

\begin{demonstration}{FormuleTaylorYoung}
    \begin{itemize}[leftmargin=0.5cm]
        \item Soit $\mathcal{H}_n$ : si $f : I \to \K$ de classe $\mathcal{C}^n$ et $a \in I$, alors : $ \strong{\dfrac{f(x) - T_{n,f,a}(x)}{(x-a)^n} \xrightarrow[x \to a]{} 0} $
        \item $\mathcal{H}_0$ est vraie : pour $f : I \to \K$ de classe $\mathcal{C}^0$ et $a \in I$, on a bien (car $f$ est continue) : 
        $$ \dfrac{f(x) - T_{0,f,a}(x)}{(x-a)^0} = f(x) - f(a) \xrightarrow[x \to a]{} 0 $$
        \item Soient $n \in \N$ tel que $\mathcal{H}_n$ soit vraie, $f : I \to \K$ de classe $\mathcal{C}^{n+1}$ et $a \in I$.\\
        Posons \strong{$\varphi : x \in I \mapsto f(x) - T_{n+1,f,a}(x)$}. D'après les théorèmes généraux, $\varphi$ est dérivable sur $I$ comme somme de fonctions dérivable et $\forall x \in I$ :
        \begin{align*}
            \varphi'(x) &= f'(x) - T'_{n+1,f,a}(x) \\
            &= f'(x) - \sum_{k=1}^{n+1} f^{(k)}(a) \dfrac{(x-a)^{(k-1)}}{(k-1)!} \\
            &= f'(x) - \sum_{k=0}^{n} f^{(k+1)}(a) \dfrac{(x-a)^k}{k!} \\
            &= \mathbox{f'(x) - T_{n,f',a}(x)}
        \end{align*}
        Soit \strong{$\varepsilon > 0$} :\\
        $f'$ est de classe $\mathcal{C}^n$ (car $f$ est de classe $\mathcal{C}^{n+1}$) donc par HR, $\exists \alpha_\varepsilon > 0, \; \forall x \in I\setminus\{a\}, \; |x-a| \leqslant \alpha_\varepsilon $ :
        $$ \left| \dfrac{f'(x) - T_{n,f',a}(x)}{(x-a)^n} \right| \leqslant \varepsilon \; \Longleftrightarrow \; \mathbox{|\varphi'(x)| \leqslant \varepsilon |x-a|^n} $$
        \begin{itemize}
            \item Si \strong{$x > a$} : alors $\forall t > a, |t-a| \leqslant \alpha_\varepsilon$ : $|\varphi'(t)| \leqslant \varepsilon(t-a)^n$\\
            D'après les \strong{IAF} appliquées à \strong{$\varphi$} et \strong{$g : t > a \mapsto \varepsilon \dfrac{(t-a)^{n+1}}{n+1}$} entre $a$ et $x$ tel que $|x-a| \leqslant \alpha_\varepsilon$ :
            $$ |\varphi(x) - \varphi(a)| \leqslant g(x) - g(a) \; \Longleftrightarrow \; |\varphi(x)| \leqslant \varepsilon \dfrac{(x-a)^{n+1}}{n+1} \; \Longleftrightarrow \; \mathbox{\left| \dfrac{f(x) - T_{n+1,f,a}(x)}{(x-a)^{n+1}} \right| \leqslant \dfrac{\varepsilon}{n+1} \leqslant \varepsilon} $$
            \item Si \strong{$x < a$} : alors $\forall t < a, |t-a| \leqslant \alpha_\varepsilon$ : $|\varphi'(t)| \leqslant \varepsilon(a-t)^n$\\
            D'après les \strong{IAF} appliquées à \strong{$\varphi$} et \strong{$g : t < a \mapsto -\varepsilon \dfrac{(a-t)^{n+1}}{n+1}$} entre $x$, tel que $|x-a| \leqslant \alpha_\varepsilon$, et $a$ :
            $$ |\varphi(a) - \varphi(x)| \leqslant g(a) - g(x) \; \Longleftrightarrow \; |\varphi(x)| \leqslant \varepsilon \dfrac{(a-x)^{n+1}}{n+1} \; \Longleftrightarrow \; \mathbox{\left| \dfrac{f(x) - T_{n+1,f,a}(x)}{(x-a)^{n+1}} \right| \leqslant \dfrac{\varepsilon}{n+1} \leqslant \varepsilon} $$
        \end{itemize}
        \item Donc $\mathcal{H}_n$ est vraie par récurrence $\forall n \in \N$.
    \end{itemize}
\end{demonstration}

\end{document}
