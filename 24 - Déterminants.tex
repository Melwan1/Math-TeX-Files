\documentclass[12pt,a4paper]{report}
\input{preambule}

\begin{document}
\chapter{Déterminants}

\section{Déterminants dans une base}

\subsection{Formes $n$-linéaires et alternées : définitions et exemples}

\begin{definition}{Forme $n$-linéaire}{}
Soit $E$ un $\K$-ev, $n \in \N^*, f : E^n \to \K$. \\
$f$ est \strong{$n$-linéaire} lorsque :
\begin{center}
$\forall (a_1,\cdots,a_n) \in E^n, \forall i \in \llbracket 1,n \rrbracket, f : x \in E \mapsto f(a_1,\cdots,a_{i-1},x,a_{i+1},\cdots,a_n)$
\end{center}
est linéaire. \\

Lorsque $n=2$, on parle de forme \strong{bilinéaire}.
\end{definition}

\begin{definition}{Application alternée}{}
$f$ est \strong{alternée} lorsque :
\begin{center}
Pour tout $(a_1,\cdots,a_n) \in E^n$, s'il existe $i \ne j$ avec $a_i = a_j$ alors $f(a_1,\cdots,a_n) = 0$.
\end{center}
\end{definition}

\begin{exemple}[Exemples]{}
\begin{enumerate}
	\item Soit $E$ un $\K$-ev, $\varphi_1,...,\varphi_n$ des formes linéaires sur $E$. \\
	Alors
	\begin{center}
		$(x_1,\cdots,x_n) \in E^n \mapsto \varphi_1(x_1)\varphi_2(x_2)\cdots \varphi_n(x_n)$
	\end{center}
	est $n$-linéaire.
	
	\item Soit $E = \K^2$, soit $f : E^2 \to \K$ bilinéaire et alternée, $\bc_2 = (e_1,e_2)$. \\
	
	Pour $x = (\alpha, \beta) = \alpha e_1+\beta e_2, \quad, y = (\gamma, \delta) = \gamma e_1 + \delta e_2$ :
	\begin{align*}
	f(x,y) &= f(\alpha e_1 + \beta e_2, \gamma e_1 + \delta e_2) &= \alpha f(e_1, \gamma e_1 + \delta e_2) + \beta f(e_2, \gamma e_1 + \delta e_2) \\
	&= \alpha (\gamma f(e_1,e_1) + \delta f(e_1,e_2)) + \beta (\gamma f(e_2,e_1) + \delta f(e_2,e_1))
	\end{align*}
	
	Ceci est vrai puisque $f(\cdot, \alpha e_1 + \beta e_2)$ est linéaire et $f(e_1,\cdot)$ et $f(\cdot, e_2)$ sont linéaires. \\
	
	$f$ est alternée donc $f(e_1,e_1) = f(e_2,e_2) = 0$. \\
	
	$f(x,y) = \alpha \delta f(e_1,e_2) + \beta \gamma f(e_2,e_1)$. \\
	
	Or $0 = f(e_1+e_2,e_1+e_2) = \underbrace{f(e_1,e_1)}_{=0} + f(e_1,e_2) + f(e_2+e_1)+\underbrace{f(e_2,e_2)}{=0}$ \\
	Donc :
	\begin{center}
		$f(e_2,e_1) = -f(e_1,e_2)$
	\end{center}
	
	D'où :
	\begin{center}
		$f(x,y) = f(e_1,e_2)(\alpha \delta - \beta \gamma)$.
	\end{center}
	
	Si on pose $\varphi(x,y) = \alpha \delta - \beta \gamma$, alors $f$ est proportionnelle à $\varphi$. \\
	
	On vérifie sans peine que $\varphi$ est bien bilinéaire et alternée (donc $k\varphi$ aussi pour tout $k \in \K$.)
\end{enumerate}
\end{exemple}

\subsection{Propriétés des formes $n$-linéaires alternées}

Soit $f : E^n \to \K$ $n$-linéaire et alternée.

\begin{proposition}{}{}
Soit $(x_1,\cdots,x_n) \in E^n$. \\
Si l'un des $x_i$ est nul, alors $f(x_1,\cdots,x_n)=0$.
\end{proposition}

\begin{demo}{}
Supposons par exemple $x_1 = 0_E$ (la même démonstration sera valable pour chacun des $x_i$, on simplifie seulement l'indexation ici). \\
$u \overset{g}{\mapsto} f(u,x_2,...,x_n)$ est linéaire de $E$ dans $\K$ donc $g(0_E) = 0$.
\end{demo}

\begin{proposition}{}{}
Soit $(x_1,\cdots,x_n) \in E^n$. Soit $i \in \llbracket 1,n \rrbracket$, $y \in \text{Vect}(x_k)_{k \in \llbracket 1,n \rrbracket \setminus \{i\}}$, alors
\begin{center}
$f(x_1,\cdots,x_i,\cdots,x_n)=f(x_1,\cdots,x_i+y,\cdots,x_n)$
\end{center}
\end{proposition}

\begin{demo}{}
Avec $i = 1$, supposons $y \in \text{Vect}(x_2,\cdots,x_n)$, on écrit $y = \displaystyle{\sum_{k=2}^n \alpha_kx_k}$. Alors :
\begin{center}
	$f(x_1+y,x_2,\cdots,x_n) = f(x_1,\cdots,x_n)+f(y,x_2,\cdots,x_n)$. \\
	et $f(y,x_2,\cdots, x_n) = f\left(\displaystyle{\sum_{k=2}^n \alpha_kx_k, x_2,\cdots,x_n}\right) = \displaystyle{\sum_{k=2}^n \alpha_k\underbrace{f(x_k,x_2,\cdots,x_n)}_{=0 \text{ car }f \text{ est alternée}}}$.
\end{center}
\end{demo}

\begin{remarque}[Variante]{}
S'il existe $i \in \llbracket 1,n \rrbracket$ tel que $x_i \in \text{Vect}(x_k)_{k \in \llbracket 1,n \rrbracket \setminus \{i\}}$ alors $f(x_1,\cdots,x_n) = 0$.
\end{remarque}

\begin{corollaire}{}{}
Si la famille $(x_1,\cdots,x_n)$ est liée, alors $f(x_1,\cdots,x_n) = 0$.
\end{corollaire}

\begin{remarque}{}
Si $E$ est de dimension finie, et $\dim E < n$, alors la seule forme $n$-linéaire alternée sur $E$ est :
\begin{align*}
E^n &\to  \K \\
(x_1,\cdots,x_n) & \mapsto 0_\K
\end{align*}
En effet, dans ce dernier cas, toute famille de vecteurs de $E$ de cardinal $n$ est liée.
\end{remarque}

\begin{proposition}{}{}
$f$ est \strong{antisymétrique}:
\begin{center}
	$\forall (x_1,\cdots,x_n) \in E^n, \forall i \ne j, f(x_1,\cdots,x_i,\cdots,x_j,\cdots,x_n) = - f(x_1,\cdots, x_j, \cdots, x_i, \cdots, x_n)$.
\end{center}
L'échange de deux variables $x_i$ et $x_j$ fait apparaître un signe $-$ dans l'expression de $f$.
\end{proposition}

\begin{demo}{}
Avec $i=1,j=2$,
\begin{align*}
0 &= f(x_1+x_2,x_1+x_2,x_3,\cdots,x_n) \\
&= \underbrace{f(x_1,x_1,x_3,\cdots,x_n)}_{=0} + f(x_1,x_2,x_3,\cdots,x_n) + f(x_2,x_1,x_3,\cdots,x_n) + \underbrace{f(x_2,x_2,x_3,\cdots,x_n)}_{=0}
\end{align*}
\end{demo}

De façon plus générale,
\begin{center}
$\forall \sigma \in S_n, \forall x_1,\cdots,x_n) \in E^n, f(x_{\sigma(1)},\cdots,x_{\sigma(n)}) = \varepsilon(\sigma) f(x_1,\cdots,x_n)$.
\end{center}

\begin{demo}{}
On vient de voir le résultat lorsque $\sigma$ est une transposition. \\

Soit $\sigma \in S_n, \sigma = \tau_1 \circ \cdots \circ \tau_r$, soit $(x_1, x_n) \in E^n$. \\
Soit $s = \tau_1 \circ \cdots \circ \tau_{r-1}$,
\begin{align*}
f(x_{\sigma(1)},x_{\sigma(2)},\cdots,x_{\sigma(n)}) &= f(x_{s (\tau_r(1))},\cdots,x_{s(\tau_r(n))}) 
&= f(y_{\tau_r(1)},\cdots,y_{\tau_r(n)})
&= -f(y_1,\cdots,y_r) = -f(x_{s(1)},\cdots,,x_{s(n)}).
\end{align*}
où $y_i = x_{s(i)}$. \\
En itérant, 
\begin{center}
$f(x_{\sigma(1)},\cdots,x_{\sigma(n)}) = (-1)^rf(x_1,\cdots,x_r)$.
\end{center}

\end{demo}

\subsection{Déterminant dans une base}

Ici, $n \in \N^*$, $E$ un $\K$-ev de dimension $n$, $\mathcal{B} = (e_1,\cdots,e_n)$ une base de $E$.

\begin{theoreme}{Définition, existence et unicité du déterminant}{DefExistenceUnicitéDéterminant}
Il existe une et une seule application $n$-linéaire alternée $f$ de $E^n$ dans $\K$ telle que 
\begin{center}
$f(e_1,\cdots,e_n) = 1$.
\end{center}
$f$ s'appelle le \strong{déterminant} dans $\mathcal{B}$ et se note $\det_{\mathcal{B}}$. \\
Toute application $n$-linéaire de $E^n$ dans $\K$ est proportionnelle à $\det_{\mathcal{B}}$.
\end{theoreme}

\begin{demo}{}
Soit $A = \mat_{\mathcal{B}}(x_1,\cdots,x_n)$. \\
$x_j = \displaystyle{\sum_{i=1}^n a_{ij}e_i}$. \\
Si $(e^*_1,\cdots,e^*_n)$ est la base duale de $\mathcal{B}$.  \\
$a_{ij} = i^{\text{ème}}$ coordonnée de $x_j$ dans $\mathcal{B} = e^*_i(x_j)$. \\

Soit $f$ convenant, 
\begin{align*}
f(x_1,\cdots,x_n) &= f\left(\sum_{i_1=1}^n a_{i_11}e_{i_1},x_2,\cdots,x_n\right) \\
&= \sum_{i_1=1}^n a_{i_11}f(e_{i_1},x_2,\cdots,x_n)\\
&= \sum_{i_1=1}^na_{i_11}f\left(e_{i_1},\sum_{i_2=1}^na_{i_22}e_{i_2},x_3,\cdots,x_n \right) \\
&= \sum_{i_1=1}^n a_{i_11} \sum_{i_2=1}^n a_{i_22} f(e_{i_1},e_{i_2},x_3,\cdots,x_n) \\
&= \sum_{1 \le i_1,i_2 \le n}a_{i_11}a_{i_22} f(e_{i_1},e_{i_2},x_3,\cdots,x_n) \\
&= \cdots \\
&= \sum_{1 \le i_1,i_2,\cdots,i_n  \le n} a_{i_11}a_{i_22}\cdots a_{i_nn}f(e_{i_1},\cdots,e_{i_n})
\end{align*}
La somme porte sur $\llbracket 1,n \rrbracket ^n$ qu'on peut identifier à $\mathcal{F}(\llbracket 1,n \rrbracket,\llbracket 1,n \rrbracket)$. \\

$f(x_1,\cdots,x_n) = \displaystyle{\sum_{s \in \mathcal{F}(\llbracket 1,n \rrbracket,\llbracket 1,n \rrbracket)} a_{s(1)1}a_{s(2)2}\cdots a_({s(n)}n) f(e_{s(1)},e_{s(2)},\cdots,e_{s(n)})}$. \\

Soit $s \in \mathcal{F}(\llbracket 1,n \rrbracket,\llbracket 1,n \rrbracket)$. \\
Si $s$ n'est pas injective, il existe $i < j$ avec $s(i) = s(j) = \ell$ \\
d'où $f(e_{s(1)},\cdots,e_{s(n)}) = f(e_{s(1)},\cdots,e_\ell,\cdots,e_\ell,\cdots,e_{s(n)}) = 0$, car $f$ est alternée. \\

On peut donc faire porter la somme sur l'ensemble des applications injectives de $\llbracket 1,n \rrbracket$ dans $\llbracket 1,n \rrbracket$ \ie sur l'ensemble $S_n$ des permutations de $\llbracket 1,n \rrbracket$. \\

$f(x_1,\cdots,x_n) = \displaystyle{\sum_{\sigma \in S_n} a_{\sigma(1)1}\cdots a_{\sigma(n)n} \underbrace{f(e_{\sigma(1)},\cdots,e_{\sigma(n)})}_{\varepsilon(\sigma) f(e_1,\cdots,e_n)}}$ \\
Donc :
\begin{center}
$f(x_1,\cdots,x_n) = f(e_1,\cdots,e_n) \displaystyle{\sum_{\sigma \in S_n} \varepsilon(\sigma) a_{\sigma(1)1}\cdots a_{\sigma(n)n}}$.
\end{center}

Si $f(e_1,\cdots,e_n) = 1$ alors :
\begin{center}
$\forall (x_1,\cdots,x_n) \in E^n, f(x_1,\cdots,x_n) = \displaystyle{\sum_{\sigma \in S_n} \varepsilon(\sigma) e^*_{\sigma(1)}(x_1) \cdots e^*_{\sigma(n)}(x_n)}$.
\end{center}

Posons donc, pour $(x_1,\cdots,x_n) \in E^n,$
\begin{center}
$\varphi(x_1,\cdots,x_n) = \displaystyle{\sum_{\sigma \in S_n} \prod_{k=1}^n e^*_{\sigma(k)}(x_k)}$
\end{center}

\begin{itemize}
	\item $\varphi(e_1,\cdots,e_n) = \displaystyle{\sum_{\sigma \in S_n} \varepsilon(\sigma) \prod_{k=1}^n e^*_{\sigma(k)}(e_k)} = \displaystyle{\sum_{\sigma \in S_n} \varepsilon(\sigma) \prod_{k=1}^n \delta_{\sigma(k)k}}$. \\
	
	Si $\sigma \ne \text{Id}$, il existe $\ell$ tel que $\sigma(\ell) \ne \ell$ donc $\delta_{\sigma(\ell)\ell} = 0$ puis
	\begin{center}
	$\displaystyle{\prod_{k=1}^n \delta_{\sigma(k)k} = 0}$
	\end{center}
	
	D'où $\varphi(e_1,\cdots,e_n) = \varepsilon(\text{Id}) \displaystyle{\prod_{k=1}^n 1} = 1$.
	
	\item $\varphi$ est $n$-linéaire. \\
	
	Soit $\sigma \in S_n$. Pour $1 \le k \le n, e^*_{\sigma(k)}$ est une forme linéaire sur $E$ \\
	donc $(x_1,\cdots,x_n) \mapsto \varepsilon(\sigma)e^*_{\sigma(1)}(x_1) \cdots e^*_{\sigma(n)}(x_n)$ est $n$-linéaire. \\
	Par somme, $\varphi$ est aussi $n$-linéaire.
	
	\item $\varphi$ est alternée. \\
	
	Soit $(x_1,\cdots,x_n) \in E^n$, supposons qu'il existe $i<j$ tel que $x_i=x_j$, montrer que $\varphi(x_1,\cdots,x_n) = 0$. \\
	
	On rappelle que $A_n$ est le groupe alterné, c'est-à-dire l'ensembles des permutations $\sigma$ paires (\ie de signature $\varepsilon(\sigma) = 1$ ou encore de nombre d'inversions pair, une inversion se présentant en un couple $(i,j) \in \llbracket 1,n \rrbracket^2$ avec $i<j$ lorsque $\sigma(i)>\sigma(j)$). \\
	Le groupe $S_n \setminus A_n$ est donc l'ensemble des permutations $\sigma$ impaires de $\llbracket 1,n \rrbracket$ (\ie de signature $\varepsilon(\sigma) = -1$). \\
	
	On a :
	\begin{center}
	$\varphi(x_1,\cdots,x_n) = \displaystyle{\sum_{\sigma \in A_n} \prod_{k=1}^n e^*_{\sigma(k)}(x_k) - \sum_{\sigma \in S_n \setminus A_n} \prod_{k=1}^n e^*_{\sigma(k)}(x_k)}$
	\end{center}
	
	Soit $\tau$ la transposition qui échange $i$ et $j$ : $\varepsilon{\tau} = -1$. \\
	Si $s$ décrit $A_n$, alors $s \circ \tau$ décrit $S_n \setminus A_n$. \\
	Donc :
	\begin{center}
	$\displaystyle{\sum_{\sigma \in S_n \setminus A_n} \prod_{k=1}^n e^*_{\sigma(k)}(x_k) = \sum_{s \in A_n} \prod_{k=1}^n e^*_{s \circ \tau(k)}(x_k)}$
	\end{center}
	
	Soit $s \in S_n$.
	\begin{itemize}
		\item Si $k \in \llbracket 1,n \rrbracket \setminus \{i,j\}$, alors $\tau(k) = k$ et $e^*_{s \circ \tau(k)}(x_k) = e^*_{s(k)}(x_k)$.
		\item $e^*_{s \circ \tau(i)}(x_i) = e^*_{s(j)}(x_i) = e^*_{s(j)}(x_j)$ (car $x_i = x_j$). \\
		$e^*_{s \circ \tau(j)}(x_j) = e^*_{s(i)}(x_j) = e^*_{s(i)}(x_i)$.
	\end{itemize}
	D'où :
	\begin{center}
	$\displaystyle{\prod_{k=1}^n e^*_{s \circ \tau(k)} (x_k)} = \displaystyle{\prod_{k=1}^n e^*_{s(k)}(x_k)}$
	\end{center}
	Donc : 
	\begin{center}
	$\displaystyle{\sum_{\sigma \in S_n \setminus A_n} \prod_{k=1}^n e^*_{\sigma(k)}(x_k) = \sum_{s \in A_n} \prod_{k=1}^n e^*_{s(k)}(x_k)}$
	\end{center}
	D'où $\varphi(x_1,\cdots,x_n) = 0$.
\end{itemize}
\end{demo}

On a vu en cours de preuve que : si $f : E^n \to \K$ est $n$-linéaire alternée, alors :
\begin{center}
$\forall (x_1,\cdots,x_n) \in E^n, f(x_1,\cdots,x_n) = f(e_1,\cdots,e_n)\varphi(x_1,\cdots,x_n)$.
\end{center}

$f$ est proportionnelle à $\varphi = \det_{\mathcal{B}}$. \\

\begin{exemple}{}
\begin{itemize}
	\item Pour $n=2$, $\mathcal{B} = (e_1,e_2)$ une base de $E$, $x,y \in E$, avec $\mat_{\mathcal{B}}(x,y) = A = \begin{pmatrix} a & c \\ b & d \end{pmatrix}$. \\
	
	Alors $\det_{\mathcal{B}}(x,y) = \displaystyle{\sum_{\sigma \in S_2} \varepsilon(\sigma) \prod_{k=1}^2 A_{\sigma(k)k}} = 1 A_{11}A_{22} + \; -1 A_{21}A_{12} = ad-bc$. 
	
	\item Pour $n = 3$, $S_3 = \{\text{Id},\tau_{12},\tau_{13},\tau_{23}, \begin{pmatrix} 1 & 2 & 3 \end{pmatrix}, \begin{pmatrix} 1 & 3 & 2 \end{pmatrix} \}$. \\
	
	Si $A = \mat_{\underbrace{(e_1,e_2,e_3)}_{\mathcal{B}}}(x_1,x_2,x_3)$, alors  
	\begin{align*}
	\det_{\mathcal{B}}(x_1,x_2,x_3) &= \sum{\sigma \in S_3} \varepsilon(\sigma) \prod_{k=1}^3 A_{\sigma(k)k} \\
	&= A_{11}A_{22}A_{33} - A_{21}A_{12}A_{33} - A_{31}A_{22}A_{13} - A_{11}A_{32}A_{23} + A_{21}A_{32}A_{13}+A_{31}A_{12}A_{23}
	\end{align*}
\end{itemize}
\end{exemple}

\end{document}