\documentclass[12pt,a4paper]{report}
\input{00 - preambule}

\begin{document}

\chapter{Fonctions continues}

\section{Définition et exemples}
\subsection{Définitions}

Dans la suite, $\mathbb{K} = \mathbb{R}$ ou $\mathbb{K} = \mathbb{C}$.  $\mathcal{D}$ est une partie non vide de $\mathbb{R}$ (très souvent, $\mathcal{D}$ est un intervalle, ou une réunion finie d'intervalles.)

\begin{definition}{Continuité d'une fonction}{}
Soit $f : \mathcal{D} \rightarrow \mathbb{K}$. \newline
$\star$ Soit $x_0 \in \mathcal{D}$. Les assertions suivantes sont équivalentes : 
\begin{enumerate}
\item \textbf{Définition topologique :} \\Pour tout voisinage $V$ de $f(x_0)$ dans $\mathbb{K}$, il existe un voisinage $U$ de $x_0$ dans $\mathbb{R}$ tel que $f(U \cap \mathcal{D}) \subset V$, soit
\begin{center}
    $\forall V \in \mathcal{V}_\mathbb{K}(f(x_0))$, $\exists U \in \mathcal{V}_\mathbb{R}(x_0)$ : $\underbrace{f(U \cap \mathcal{D}) \subset V}_{\forall x \in U\cap \mathcal{D}, f(x) \in V}$ 
\end{center}


\item Pour tout $\varepsilon > 0$, il existe $\alpha > 0$ tel que
\newline $\forall x \in \mathcal{D}, \lvert x-x_0 \rvert \leq \alpha \Longrightarrow \lvert f(x)-f(x_0) \leq \varepsilon$, soit
\begin{center}
    $\forall \varepsilon > 0, \exists \alpha > 0, \forall x \in \mathcal{D}, \lvert x-x_0 \rvert \leq \alpha \Longrightarrow \lvert f(x)-f(x_0) \leq \varepsilon$
\end{center}


\item \textbf{Définition séquentielle :} \\Pour toute suite $(u_n)_{n \in \mathbb{N}}$ de points de $\mathcal{D}$ qui converge vers $x_0$, la suite $(f(u_n))_{n \in \mathbb{N}}$ converge vers $f(x_0)$.

\end{enumerate}


\noindent $\star \star$ On dit que $f$ est continue sur $\mathcal{D}$ si $f$ est continue en tout point de $\mathcal{D}.$
\end{definition}

\begin{demo}{}
\textbf{1) $\Longrightarrow 2)$}\\
Soit $\varepsilon > 0$, $V=\lbrace z\in K \mid \lvert f(x) -z \rvert \leq \varepsilon \rbrace$ est un voisinage de $f(x_0)$ dans $K$ \\
On peut donc trouver $U \in \mathcal{V}_\R (x_0)$ tel que $f(U\cap D) \subset V$ \\
$U$ est voisinage de $x_0$ dans $\R$ donc il existe $\alpha > 0$ tel que $[x_0 - \alpha, x_0 + \alpha ]\subset U$\\
\\
Soit $x\in \mathcal{D}$ tel que $\lvert x-x_0 \rvert \leq \alpha$, on a alors $x\in [x_0 - \alpha , x_0 + \alpha] \subset U$ \\
D'où $x\in U \cap D$ donc $f(x) \in V$, autrement dit : $\lvert f(x) - f(x_0)\rvert \leq \varepsilon$
\\
\textbf{2) $\Longrightarrow$ 3)}\\
Soit $(U_n)$ une suite de points de $\mathcal{D}$ tel que $U_n \xrightarrow[n \rightarrow + \infty] {} x_0$\\
Montrons que : $f(U_m) \xrightarrow[n \rightarrow + \infty]{} f(x_0)$\\
Soit $\varepsilon > 0$, on dispose de $\alpha > 0$ tel que, pour tout $x$ dans $\mathcal{D}$ : 
\begin{center}
    $\lvert x-x_0 \rvert \leq \alpha \Longrightarrow \lvert f(x) - f(x_0) \rvert \leq \varepsilon$
\end{center}
$u$ converge vers $x_0$ donc il existe $n_0 \in \N$ tel que : 
\begin{center}
    $\forall n \geq n_0$, $\lvert U_n - x_0 \rvert \leq \alpha$
\end{center}
Soit $n\geq n_0$, on a : $U_n \in \mathcal{D}$ et $\lvert U_n -x_0 \rvert \leq \alpha$\\
Donc $ \lvert f(U_n) - f(x_0) \rvert \leq \varepsilon$, on a bien vérifié : 
\begin{center}
    $\forall \varepsilon > 0$, $\exists n_0 \in \N$, $\forall n \geq n_0$, $\lvert f(U_n) - f(x_0)\rvert \leq \varepsilon$
\end{center}
\textbf{3) $\Longrightarrow$ 1)}\\
\Strong{Par contraposée} \\
Supposons non \textbf{1)} : $\exists V \in \mathcal{V}_{K} (f(x_0)), \forall U \in \mathcal{V}_{\R} (x_0), \exists x \in \mathcal{D} \cap U, f(x) \notin V$\\
En particulier, si $n\in \N^*$ avec $U_n = [x_0 - \frac{1}{n}, x_0 + \frac{1}{n}]$, on a $U_n \in \mathcal{V}_{\R} (x_0)$. \\ Donc il existe $U_n \in \mathcal{D}\cap U_n$ tel que $f(U_n) \notin V$\\
$(U_n)_{n \in \N^*}$ converge vers $x_0$ ( car $\forall n\geq 1, \lvert U_n - x_0\rvert \leq \frac{1}{n}$) mais $(f(U_n))$ ne converge pas vers $f(x_0)$, car pour tout $n$, $f(U_n) \notin V$\\
\\
$\Longrightarrow$ Assertion non \textbf{3)}
\end{demo}

\subsection{Exemples}
Si $\lambda \in \mathbb{K}$ alors $x \in \mathcal{D} \mapsto \lambda$ est continue.

\begin{demo}
\underline{Preuve} : Soit $x_0 \in \mathcal{D}, \varepsilon > 0$, soit $\alpha > 0$ quelconque.
\newline Pour $x \in \mathcal{D}$ tel que $\lvert x-x_0 \rvert \leq \alpha$, on a $\lvert f(x)-f(x_0) \rvert = \lvert \lambda - \lambda \rvert = 0 \leq \varepsilon$
\end{demo}


\subsubsection{Fonctions lipschitziennes}
\begin{definition}{Fonction lipschitzienne}{}
$f : \mathcal{D} \rightarrow \mathbb{K}$ est lipschitzienne s'il existe $k \in \mathbb{R}_+$ tel que :
\begin{center}
    $\forall x,y \in \mathcal{D}, \lvert f(x)-f(y) \rvert \leq k \lvert x-y \rvert$.
\end{center}

On dit alors que $f$ est $k$-lip.
\end{definition}

\begin{exemple}[Exemple : Fonctions lipschitziennes]
\begin{enumerate}
\item Pour $\alpha,\beta \in \mathbb{K}, t \in \mathcal{D} \mapsto \alpha t + \beta$ est $\lvert \alpha \rvert$-lip.
\item $\sin$ et $\cos$ sont $1$-lip.
\end{enumerate}
\end{exemple}

\begin{proposition}{Lien entre fonction lipschitzienne et continue}{}
Si $f$ est lipschitzienne alors $f$ est continue.
\end{proposition}

\begin{demo}{}
Soit $x_0 \in \mathcal{D}$, $\varepsilon > 0$ \\
Soit $k>0$ tel que $\forall x,y \in \mathcal{D}$, $\lvert f(x)-f(y)\rvert \leq k\lvert x-y\rvert$ et posons $ \alpha = \frac{\varepsilon}{k}$\\
Alors, si $x\in \mathcal{D}$ est tel que $ \lvert x-x_0 \rvert \leq \alpha $, on a $\lvert f(x) -f(x_0) \rvert \leq k\lvert x-x_0 \rvert$ $\leq k\alpha = \varepsilon$\\
$\Longrightarrow f$ est continue en $x_0$
\end{demo}

\begin{remarque}
Soit $f : t \mapsto t^2 $(dans $\mathbb{R}$). $f$ est continue : 
\newline Soit $x_0 \in \mathbb{R}, (u_n)$ une suite de réels qui converge vers $x_0$. \newline D'après les théorèmes généraux, $u_n^2 \xrightarrow[n \rightarrow +\infty]{} x_0^2 = f(x_0)$, donc $f$ est continue en $x_0$.
\end{remarque}


\subsubsection{Fonctions uniformément continues}
\begin{definition}{Fonction uniformément continue}{}
$f : \mathcal{D} \rightarrow \mathbb{K}$ est uniformément continue si : \newline $\forall \varepsilon > 0, \exists \alpha_\varepsilon > 0, \forall x,y \in \mathcal{D}, \lvert x-y \rvert \leq \alpha_\varepsilon \Longrightarrow \lvert f(x)-f(y) \rvert \leq \varepsilon$
\end{definition}

\begin{remarque}
$f$ lipschitzienne $\Longrightarrow f$ uniformément continue
\end{remarque} 

\begin{exemple}[Exemple : Étude de la fonction racine]
Soit $f : x \rightarrow \sqrt{x}$ (dans $\mathbb{R}_+$)
\newline \textbf{Rappel :} $\forall u,v \in \mathbb{R}_+, \lvert \sqrt{u} - \sqrt{v} \rvert \leq \sqrt{\lvert u-v \rvert}$.
\newline Soit $\varepsilon > 0$. Posons $\alpha = \varepsilon^2$. \newline Pour $x,y \in \mathbb{R}_+$ tel que $\lvert x-y \rvert \leq \alpha$, on a 
$\lvert \sqrt{x}-\sqrt{y}  \rvert \leq \sqrt{\lvert x-y \rvert} \leq \sqrt{\alpha} \leq \varepsilon$. 
\newline \newline
Supposons $f$ lipschitzienne, soit $k \in \mathbb{R}$ tel que : \newline $\forall x,y \in \mathbb{R}_+, \lvert \sqrt{x} - \sqrt{y} \rvert \leq k \lvert x-y \rvert$. \newline
En particulier, pour $x > 0, \sqrt{x} \leq kx$ ie $\dfrac{1}{\sqrt{x}} \leq k$, ceci est faux pour $x = \dfrac{1}{(k+1)^2}$
\end{exemple}

On a $f$ uniformément continue $\Longrightarrow f$ continue. \newline

\begin{remarque}
\textbf{Réciproque fausse : } Soit $f : t \in \mathbb{R}_+ \longrightarrow t^2$,  et pour $n \in \mathbb{N}^*, x_n = n$ et $y_n = n + \frac{1}{n}$
\noindent On a $y_n - x_n \xrightarrow[n \rightarrow +\infty]{} 0$ et $f(y_n) - f(x_n) = 2 + \frac{1}{n^2} \geq 2$.
\newline Soit $\varepsilon = 1$. Si $f$ est uniformément continue, il existe $\alpha > 0$ tel que $\forall u,v \in \mathbb{R}_+, \lvert u-v \rvert \leq \alpha \Longrightarrow \lvert f(x) - f(y) \rvert \leq 1$.
\newline Or, pour $n$ assez grand, $\lvert y_n - x_n \rvert \leq \alpha$ mais $\lvert f(y_n) - f(x_n) \rvert \geq 2 > 1$.
\end{remarque}


\textbf{Bilan : } $f : \mathcal{D} \rightarrow \mathbb{K}$ : $f$ lipschitzienne $\Longrightarrow f$ uniformément continue $\Longrightarrow f$ continue (et aucune des réciproques n'est vraie).
\newline \newline On a une réciproque partielle :

\begin{theoreme}{Théorème de Heine}{}
On suppose que $\mathcal{D}$ est fermé et borné (c'est le cas par exemple si $D = \left[a,b\right]$ avec $a,b \in \mathbb{R}, a < b$). Soit $f : \mathcal{D} \rightarrow \mathbb{K}$ continue. Alors $f$ est uniformément continue.
\end{theoreme}

\begin{principedemo}{}

\end{principedemo}


\begin{proposition}{Petite Histoire - Continuité et voisinages}{}
Soit $f : \mathcal{D} \rightarrow \mathbb{R}$ continue en $x_0 \in \mathcal{D}$.
Si $\alpha < f(x_0)$ alors $f$ est à valeurs supérieures à $\alpha$ au voisinage de $x_0$. \newline
Il existe un voisinage de $x_0$ tel que : $\forall x \in \mathcal{U} \cap \mathcal{D}, f(x) \geq \alpha$.
\newline (Il existe donc $r > 0$ tel que $\forall x \in \mathcal{D}, \lvert x-x_0 \rvert \leq r \Longrightarrow f(x) \geq \alpha$ : en effet, il suffit de choisir $r > 0$ tel que $\left[x_0-r,x_0+r\right] \in \mathcal{U}$)
\end{proposition}

\begin{principedemo}{}

\end{principedemo}

\begin{remarque}
Donc, si $f(x_0) > 0$, on a en choisissant $m \in \left] 0, f(x_0) \right[ : f$ est à valeurs supérieures à $m$ donc strictement positives au voisinage de $x_0$ : 
\newline $\exists r > 0, \forall x \in \mathcal{D}, \lvert x-x_0 \rvert \leq r \Longrightarrow f(x) \geq m > 0$. \newline
De même, si $\beta > f(x_0)$, on aura $f$ à valeurs inférieures à $\beta$ au voisinage de $x_0$ (car $\left] -\infty, \beta \right[$ est alors au voisinage de $f(x_0)$). \newline En particulier, si $f(x_0) < 0$, et si on prend un $M \in \left] f(x), 0 \right[$, alors $f(x) \leq M < 0$ au voisinage de $x_0$.\\
\\
\end{remarque}


\section{Théorèmes généraux}

\begin{theoreme}{Opérations}{}
Soit $f, g : \mathcal{D} \rightarrow \mathbb{K}, x_0 \in \mathcal{D}, \lambda \in \mathbb{K}$. On suppose $f,g$ continues en $x_0$. Alors :
\begin{enumerate}
\item $\lambda f + g$     , $fg$     et $\frac{1}{f}$ (si $f$ ne s'annule pas) sont continues en $x_0$.
\item Si $\mathbb{K}=\mathbb{R}$, alors $\rm{inf}(f,g) ,  \rm{sup}(f,g) , f^+ , f^-$ sont continues en $x_0$.
\item Si $\mathbb{K}=\mathbb{C}$, alors $\Re(f)$,$\Im(f)$, $\overline{f}$ sont continues en $x_0$.
\end{enumerate}
\end{theoreme} 

\pagebreak

\begin{theoreme}{Extension à un domaine}{}
Soit $f, g : \mathcal{D} \rightarrow \mathbb{K}, \lambda \in \mathbb{K}$. On suppose $f,g$ continues sur $\mathcal{D}$. Alors :
\begin{enumerate}
\item $\lambda f + g$     , $fg$     et $\dfrac{1}{f}$ (si $f$ ne s'annule pas) sont continues sur $\mathcal{D}$.
\item Si $\mathbb{K}=\mathbb{R}$, alors $\rm{inf}(f,g) ,  \rm{sup}(f,g) , f^+ , f^-$ sont continues sur $\mathcal{D}$.
\item Si $\mathbb{K}=\mathbb{C}$, alors $\Re(f)$,$\Im(f)$, $\overline{f}$ sont continues sur $\mathcal{D}$.
\end{enumerate}
\end{theoreme} 

\begin{tcolorbox}[colback=gray!5!white,colframe=gray!75!black,title=Parenthèse - Opérations sur les fonctions à valeurs dans $\mathbb{R}$ ou $\mathbb{C}$]
Soit $X$ un ensemble non vide. Pour $f,g \in \mathcal{F}(X,\mathbb{K})$ et $\lambda \in \mathbb{K}$.
\newline $f+g$ est l'application $x \in X \mapsto f(x) + g(x)$,
\newline $\lambda f$ est l'application $x \in X \mapsto \lambda f(x)$,
\newline $fg$ est l'aplication $x \in X \mapsto f(x)g(x)$,
\newline $\lvert f \rvert$ est l'application $x \in X \mapsto \lvert f(x) \rvert$,
\newline Si $f$ ne s'annule pas, $\dfrac{1}{f}$ est l'application $x \in X \mapsto \dfrac{1}{f(x)}$,
\begin{enumerate}
\item Si $\mathbb{K} = \mathbb{R}$,
\par $\rm{inf}(f,g)$ est l'application $x \in X \mapsto \rm{min}\left(f(x),g(x)\right)$,
\par $\rm{sup}(f,g)$ est l'application $x \in X \mapsto \rm{max}\left(f(x),g(x)\right)$,
\par $f^+$ est l'application $x \in X \mapsto \rm{max}\left(f(x),0\right)$,
\par $f^-$ est l'application $x \in X \mapsto - \rm{min}\left(f(x),0\right)$,
\item Si $\mathbb{K} = \mathbb{C}$,
\par $\Re(f)$ est l'application $x \in X \mapsto \Re\left(f(x)\right)$,
\par $\Im(f)$ est l'application $x \in X \mapsto \Im\left(f(x)\right)$,
\par $\overline{f}$ est l'application $x \in X \mapsto \overline{f(x)}$
\par On a : $f = \Re(f) + i \Im(f)$
\par $\Re(f) = \dfrac{f + \overline{f}}{2}$
\par $\Im(f) = \dfrac{f - \overline{f}}{2i}$

\end{enumerate}
\end{tcolorbox}

Le théorème 2.1 se démontre très facilement grâce à la définition séquentielle.\\

\begin{principedemo}{}

\end{principedemo}

\begin{exemple}[Exemples : Continuité d'un quotient de fonctions polynomiales]{}
On sait que $x \in \mathcal{D} \mapsto x$ est continue, et même lipschitzienne. Donc, par produit : \newline $\forall k \in \mathbb{N}, x \in \mathcal{D} \mapsto x^k$ est continue. \newline Puis, pour $\lambda \in \mathbb{K}, x \in \mathcal{D} \mapsto \lambda x^k$ est continue.
\newline Par somme, si $P \in \mathbb{K}\left[X\right]$, alors $t \mapsto \widetilde{P}(t)$ est continue sur $\mathcal{D}$.
\newline Par quotient, si $Q \in \mathbb{K}\left[X\right]$ est tel que $\widetilde{Q}(t) \neq 0$ pour $t \in \mathcal{D}$ alors  $t \mapsto \dfrac{\widetilde{P}(t)}{\widetilde{Q}(t)}$ est continue sur $\mathcal{D}$
\end{exemple}

\begin{theoreme}{Composition}{}
Soit $f : \mathcal{D} \rightarrow \Delta \subset \mathbb{R}$ continue, \newline $g : \Delta \rightarrow \mathbb{K}$ continue.
\newline Alors $g \circ f$ est continue.

\end{theoreme}

\begin{principedemo}{}

\end{principedemo}

\begin{proposition}{Composition par l'exponentielle complexe}{}
Soit $f : \mathcal{D} \rightarrow \mathbb{C}$ continue. Alors $\rm{exp} \circ f$ est continue.
\end{proposition}

\begin{remarque}
\begin{enumerate}
\item $g : \mathcal{D} \rightarrow \mathbb{C}$ est continue $\Longleftrightarrow \Re(g)$ et $\Im(g)$ sont continues ;
\item $\rm{ln}$ et $t \mapsto e^t$ (exp réelle) sont continues. $\sin$ et $\cos$ aussi.
\end{enumerate}
\end{remarque}

\section{Théorème des valeurs intermédiaires}
\begin{theoreme}{Version n°1}{}
Soit $I$ intervalle non vide de $\mathbb{R}$, $f:I\longrightarrow\mathbb{R}$
continue. Alors $f\left(I\right)$ est un intervalle.
\end{theoreme}

\begin{theoreme}{Version n°1 bis}{}
Soit $I$ un intervalle de $\R$, $f:I \rightarrow \R$ continue \\
Si $\alpha$ et $\beta$ sont 2 valeurs prises par $f$ alors tout réel compris entre $\alpha$ et $\beta$ est aussi une valeur prise par $f$.
\end{theoreme}

\begin{theoreme}{Version n°2}{}
Soient $a,b\in\mathbb{R}$ avec $a<b$ et $f:\left[a,b\right]\longrightarrow\mathbb{R}$
continue. Si $f(a)f(b)\leqslant0$, alors il existe un point $c \in [a,b]$ tel que $f(c)=0$.\\
(si $f$ prend des valeurs positives et des valeurs négatives alors $f$ s'annule)
\end{theoreme}

\begin{demo}
\begin{itemize}
    \item \textbf{Version n°1 $\Longleftrightarrow$ Version n°1 bis } \\
    $\forall \alpha, \beta \in f(I)$, $[\alpha, \beta] \subset f(I)$ \\
    \ie $f(I)$ est une partie convexe de $\R$, autrement dit un intervalle \\
    \\
    \item \textbf{Version n°1 $\Longleftrightarrow$ Version n°2 } \\
    $\Longrightarrow$ $[a,b]$ est un intervalle donc $f([a,b])$ aussi donc : $[f(a),f(b)]\subset f([a,b])$\\
    Or, si $f(a)f(b) \leq 0$ alors $0\in [f(a), f(b)]$, donc $0\in f([a,b])$\\
    \\
    $\Longleftarrow$ Soient $\alpha, \beta \in f(I)$ avec $\alpha < \beta$. Montrons que : $[\alpha, \beta] \subset f(I)$ (On montre que $f(I)$ est convexe de $\R$)\\
    Soit $\gamma \in [\alpha, \beta]$, on dispose de $a,b \in I$ avec $f(a)=\alpha , f(b)=\beta$\\
    On a par exemple : $a<b$\\
    Soit $g:t \in [a,b] \rightarrow f(t) - \gamma \in \R$ \\
    $g$ est continue car $f$ l'est : $g(a)g(b)= \underbrace{(\alpha - \gamma)}_{\leq 0} \underbrace{(\beta - \gamma)}_{\geq 0} \leq 0$\\
    D'après la \textbf{Version n°2}, $g$ s'annule : il existe $c\in [a,b]$ tel que $g(c)=0$ \ie $f(c)=\gamma$
    
\end{itemize}
\end{demo}

\begin{demo}
\textbf{Preuve de Version n°2}\\
\textbf{Méthode 1 : Coup de coeur de O. S. au lycée St-L.}\\
\\
\textbf{Méthode 2 : Dichotomie (Classique)}\\
\end{demo}

\begin{application}{Un marcheur}{}
Un marcheur parcourt $20 km$ en $4h$.\\ 
Montrons qu'il y a un intervalle de temps de $1h$ pendant lequel il a parcouru $5 km$.\\
\end{application}

\begin{application}{Exercice}{}
$a<b$, $f:[a,b] \rightarrow [a,b]$ continue \\
Montrer que $f$ a un point fixe (\ie il existe $x\in [a,b]$ tel que $f(x) =x$)
\end{application}

\begin{remarque}[Remarques]
\begin{itemize}
    \item $E = $ partie entière \\
    $E(\R)=\Z$. $\Z$ n'est pas un intervalle donc $t \mapsto E(t)$ n'est pas continue.\\
    On en déduit $t \mapsto Frac(t)$ n'est pas continue. 
    \item $D=\R^*$\\
    $f:D \rightarrow \R$\\
    $t\mapsto \left\{\begin{array}{ll} 0 & \mbox{si} \mbox{ } t<0 \\ 1 & \mbox{si} \mbox{ } t>0
    \end{array}\right.$ \\
    $f$ est continue mais $f(D)= \lbrace 0,1 \rbrace$ n'est pas un intervalle, ici $D$ n'est pas un intervalle. 
\end{itemize}
\end{remarque}

\pagebreak
 
\section{Continuité et fermés bornés}
\begin{theoreme}{}{}
Soit $D$ une partie fermée bornée de $\R$, \\
$f:D \mapsto \K$ continue, alors $f(D)$ est un fermé borné dans $\K$
\end{theoreme}

\begin{demo}

\end{demo}

\begin{application}{}{}
Soient $a,b \in \R$, $f:[a,b] \rightarrow \R$ continue, alors $f$ est bornée et atteint ses bornes :\\
il existe $c,d \in [a,b]$ tel que $\forall x \in [a,b], f(c) \leq f(x) \leq f(d)$
\end{application}

\begin{demo}
$[a,b]$ est un fermé borné de $\R$ donc $f([a,b])$ est une partie fermée et bornée de $\R$ \\
$\alpha = \inf f([a,b])$ et $\beta = \sup f([a,b])$\\
$\alpha$ est adhérent à $f(D)$ donc appartient à $f(D)$ car $f(D)$ est fermé : \\
il existe $c \in [a,b], \alpha = f(c)$, de même, $\exists d \in [a,b], \beta = f(d)$
\end{demo}

\begin{remarque}
Ce théorème est valable pour $f : D \rightarrow \R$ avec $D$ une partie fermée bornée de $\R$
\end{remarque}

\begin{theoreme}{Image d'un segment par une application continue réelle}{}
$f : [a,b] \rightarrow \R$ continue ($a<b$)\\
Alors $f([a,b])$ est un segment 
\end{theoreme}

\begin{demo}
D'après le théorème précédent, il existe $c, d \in [a,b]$ tel que : \\
\begin{center}
    $\forall x \in [a,b]$, $f(c) \leq f(x) \leq f(d)$
\end{center}
Donc $f([a,b])\subset [f(c),f(d)]$\\
$[a,b]$ est un intervalle donc \strong{TVI}, $f([a,b])$ aussi \\
Or $f(c) \in f([a,b])$ et $f(d) \in f([a,b])$\\
Donc $[f(c),f(d)] \subset f([a,b])$
\end{demo}

\begin{remarque}[Attention !!!]

\end{remarque}

\end{document}
